\documentclass[A4,11pt,oneside]{book}
\usepackage[margin=1.15in]{geometry}
\textheight=19.94cm 
\textwidth=15.5cm 
%\topmargin=-4cm 
%\bottomargin=-2cm 
%\oddsidemargin=-0.2 cm 
%\evensidemargin=2 cm 
%\renewcommand{\baselinestretch}{1.5} 
%\parindent=1.5cm 
\usepackage{tikz}
\usepackage{textcomp}
\usetikzlibrary{matrix,arrows,decorations.pathmorphing}
\usepackage{amsmath,amscd}
\usepackage{amssymb}
\usepackage{calrsfs}
\usepackage{amsthm}
\usepackage[all,cmtip]{xy}

\usepackage[toc,page]{appendix}

%\usepackage[spanish,es-lcroman]{babel}
%\usepackage[english, es-lcroman]{babel}

\usepackage{courier}

\usepackage{dsfont}

\usepackage{enumerate}
\usepackage{esint}

\usepackage{float}
\usepackage[T1]{fontenc}

\usepackage{graphicx}

\usepackage[utf8x]{inputenc}

\usepackage{listings}

\usepackage{mathrsfs}
\usepackage{mathtools}

\usepackage{stmaryrd}

\usepackage{textcomp}
\usepackage{pdfpages}

\pagestyle{headings}

\theoremstyle{definition}

\newtheorem{teo}{Theorem}
\newtheorem{defi}{Definition}
\newtheorem{ejem}{Examples}
\newtheorem{lema}{Lemma}
\newtheorem{prop}{Proposition}
\newtheorem{demo}{Proof}

\begin{document}

% \addcontentsline{toc}{chapter}{Introducci\'on}{}
        \newpage

%  \includepdf[pages={1}]{cover.pdf}

%\chapter{Steenrod algebra (Masher and Tangora)}

%\section{Introduction}
%{\LARGE\textbf{Steenrod algebra (Masher and Tangora)}}\\

%A cohomology operation is a natural transformation $\theta:H^m(X;G)\rightarrow H^n(X;H)$, that is for all maps $f:X\rightarrow Y$ the following diagram
%\begin{displaymath}
%\xymatrix{ 
%H^m(X;G)\ar[r]^{\theta_X}\ar[d]^{f^*} &H^m(X;G')\ar[d]^{f^*} \\
%H^m(Y;G)\ar[r]^{\theta_Y} &H^m(Y;G')
%}
%\end{displaymath}
%commutes.
%\begin{defi}
%We denote by $K(G, n)$ any space which has only one non-trivial homotopy group, namely $\pi_n(K(G,n))=G$. 
%\end{defi}
%The Hurewicz homomorphism $h:\pi_i(X)\rightarrow H_i(X)$ is defined for any $X$ and any $i$ by choosing a generator $u$ of $H_i(S^i)$ and putting $h:[h]\rightarrow f_*(u)$ where $f:S^i\rightarrow X$. 

%\begin{defi}
%A space $X$ is said to be $n$-connected if $\pi_i(X)$ is trivial for all $i\leq n$. 
%\end{defi} 
%The Hurewicz theorem states that if $X$ is $(n-1)$-connected, then the Hurewicz homomorphism is an isomorphism in dimensions $ i \leq n$ and is still an epimorphism in dimension $n+1$. This theorem is modified if
%$n=1$; in this case the epimorphism $h:\pi_1(X)\rightarrow H_1(X)$ has a kernel the commutator subgroup of $\pi(X)$ and $h$ does not necessarily maps $\pi_2(X)$ onto $H_2(X)$.\\
%The universal coefficient theorem for cohomology gives an exact sequence
%\begin{equation*}
%0\rightarrow {\rm{Ex}}t(H_{n-1}(X),G)\rightarrow H^n(X;G)\rightarrow {\rm{Hom}}(H_n(X),G)\rightarrow 0
%\end{equation*}
%for any space X. If $X$ is $(n-1)$-connected, this becomes an isomorphism between the last two terms of the sequence, since Ext$(0,G)=0$. Now if $G=\pi_n(X)$ then the group Hom$(H_n(X),\pi_n(X))$ contains $h^{-1}$, the inverse of the Hurewicz homomorphism, which is also an isomorphism.
%\begin{defi}
%Let $X$ be $(n-1)$-connected. The fundamental class of $X$ is the cohomology class $\iota\in H^n(X;\pi_n(X))$ which correspond to $h^{-1}$ under the above isomorphism.We write sometimes $\iota_n$ or $\iota_X$ instead. 
%\end{defi} 
%\begin{teo}
%There is a one-to-one correspondence $[X,K(G,n)]\rightarrow H^n(X;G)$, given by $[f]\rightarrow f^*(\iota_n)$.
%\end {teo}
%Notation: let's write $H^m(G,n;G')$ for $H^m(K(G,n);G')$. 

%\begin{teo}
%Let denote $O(n,G; m,G')$ the set of all cohomolgy operations. There is a one-to one correspondence 
%\begin{equation*}
%O(G,n;G',m)\rightarrow H^m(G,n;G')
%\end{equation*}
%given by $\theta\rightarrow\theta(\iota_n)$.
%\end{teo}

%%%%%%%%%%%%%%%%%%%%%%%%%%%%%%%%%%%%%%%%%%%%%%%%
%%%%%%%%%%%%%%%%%%%%%%%%%%%%%%%%%%%%%%%%%%%%%%%%%Read about Obstruction theory!!!


%\section{Steenrod squares}

%Steenrod squares are cohomology operation of type $(\mathbb {Z}_2,n;\mathbb{Z}_2,n-i)$. Given an abelian group $G$ it is possible to construct a $CW$-complex $K(G,n)$ where $n\geq 2$. In particular, we can construct a complex $K(\mathbb {Z}_2,1)$.

%\begin{defi}
%For each integer $i\geq 0$, define a \emph{cup-i product}
%\begin{equation*}
%C^p(K)\otimes C^q\rightarrow C^{p+q-i}(K):(u,v)\rightarrow u\smile_i v
%\end{equation*}
%\end{defi}
% where $C^p(X)$.

\chapter{Fibre bundles and Classifying Spaces}

\begin{defi}
Let $B$ be a pointed topological space. A (locally trivial) fibre bundle over $B$ consists of a map $p:E\rightarrow B$ such that for all $b\in B$ there exists an open neighbourhood $U$ of $b$ for which there is a homeomorphism $\phi:p^{-1}(U)\rightarrow p^{-1}(*)\times U$ satisfying 
\begin{equation*}
\pi''\circ\phi=p\bigl |_U
\end{equation*}
where $\pi''$ denotes projection onto the second factor. 
\end{defi}
In this case we say that $B$ is  the base space of the fibre bundle, $X$ the total space, $p^{-1}$ the fibre, and $p$ the projection. A bundle map of fibre bundles consists of maps between the total spaces and base spaces which form a commutative square with the bundle projections. 
\begin{defi}
Let $G$ be a topological group and $B$ a topological space. A \emph{principal G-bundle} over $B$ consists of a fibre bundle $p:X\rightarrow B$ together with an action $G\times X\rightarrow X$ such that:
\begin{enumerate}
\item the  map $G\times X\rightarrow X\times X$ given by $(g,x)\mapsto (x,g\cdot x)$ maps $G\times X$ homeomorphically to its image;
\item $B= X/G$ and $p:X\rightarrow X/G$ is the quotient map;
\item for all $b\in B$ there exists a open neighbourhood $U$ of $b$ such that $p:p^{-1}(U)\rightarrow U$ is $G$-bundle isomorphic to the trivial bundle $\pi":G\times U\rightarrow U$. That is, there exists a homeomorphism $\phi:p^{-1}(U)\rightarrow G\times U$ satisfying $p=\pi"\circ\phi$ and $\phi(gx)=g\phi(x)$, where $g(g',u)=(gg',u)$.
\end{enumerate}
\end{defi}
Let $\xi$ be a principal $G$-bundle $p:X\rightarrow G$. Given a map $f:B'\rightarrow B$, the pullback yields a principal $G$-bundle over $B'$, written $f^{*}(\xi)$, and the pullback square becomes a bundle map from $f^{*}(\xi)$ to $\xi$.

\begin{defi}
A (numerable) principal $G$-bundle $\gamma$ over a pointed space $\tilde B$ is called a \emph{universal G-bundle} if 
\begin{enumerate}
 \item for any (numerable) principal $G$-bundle $\xi$ there exists a map $f:B\rightarrow\tilde B$ from the base space $B$ of $\xi$ to the base space of $\gamma$ such that $\xi=f^{*}(\gamma)$;
 \item whenever $f$,$h$ are two pointed maps from some space $B$ into the base space $\tilde B$ of $\gamma$ such that $f^{*}(\gamma)\cong h^{*}(\gamma)$ then $f\simeq h$.
 \end{enumerate}
\end{defi}

In other words, a numerable principal $G$-bundle $\gamma$ with base space $\tilde B$ is a universal $G$-bundle if, for any pointed space $B$, pullback induces a bijection from $[B,\tilde B]$ to isomorphism classes of numerable principal bundles over B. 

Let $G$ be a topological group. Let $EG$ be the infinite join $EG=\underset{\overrightarrow{n}}{\rm{lim}}\;G^{*n}$. Explicity, as a set 
\begin{equation*}
EG=\{(g_0,t_0,g_1,t_1,\dots,g_n,t_n,\dots)\in(G\times I)^{\infty}\}/\sim
\end{equation*}
such that at most finitely many $t_i$ are nonzero, $\sum t_i=1$, and
\begin{equation*}
(g_0,t_0,g_1,t_1,\dots,g_n,0,\dots)\sim(g_0,t_0,g_1,t_1,\dots,g'_n,0,\dots).
\end{equation*}
A $G$-action on $EG$ is given by
\begin{equation*}
g\cdot(g_0,t_0,g_1,t_1,\cdots,g_n,t_n,\cdots)=(gg_0,t_0,gg_1,t_1,\cdots,gg_n,t_n,\cdots).
\end{equation*}
Let $BG=EG/G$. We also write $E_nG=G^{*(n+1)}$ and $B_n(G)=E_nG/G$, referring to the inclusions $B_0G\hookrightarrow B_1G\hookrightarrow B_2G \hookrightarrow\dots B_nG\hookrightarrow\dots $ as the Milnor filtration on $BG$.
Since homotopy classes of maps into $BG$ classify principal $G$-bundles, $BG$ is called the classifying space of the group $G$. 
\begin{teo}
For every topological group $G$, the quotient map $EG\rightarrow BG$ is a (numerable) $G$-bundle and this bundle is a universal $G$-bundle.
\end{teo}

\begin{defi}
Let $p:P\rightarrow B$ be a principal $G$-bundle. The gauge group Aut(P) of $P$ is the subspace of all G-equivariant maps $u\in{\rm{Map(P,P)}}$ such that  the following diagram
\begin{displaymath}
\xymatrix{ 
P\ar[r]^u\ar[d]^{p} &P\ar[d]^{p} \\
B\ar[r] &B
}
\end{displaymath}

 commutes.
\end{defi}

\begin{defi}
Let $\xi$ be a principal $SO(n)$-bundle over an oriented manifold $B$. A spin structure on $\xi$ is a pair $(\eta,F)$ consisting of 
\begin{enumerate}
\item A principal Spin(n)-bundle $\eta$ over $B$
\item A map $f:E(\eta)\rightarrow E(\xi)$ such that the following diagram is commutative .
\end{enumerate}
\begin{displaymath}
\xymatrix{ 
E(\eta)\times Spin(n) \ar[r]\ar[d]^{f\times\lambda} &E(\eta)\ar[d]^{f}\ar[r]& B\ar@{=}[d]\\
E(\eta)\times SO(n)\ar[r] &E(\xi)\ar[r]& B
}
\end{displaymath}
Here $\lambda$ denotes the standard homomorphism from Spin (n) to SO(n).
\end{defi}

\chapter{The topology of the gauge group}

%\begin{prop}
%Let $Aut(P)$ denote the group of $G-$equivariant maps of $P$, which cover the identity map of $X$. Then
%\begin{equation*}
%\mathcal{G}\simeq Aut(P).
%\end{equation*}
%\end{prop}
%\begin{proof}
%Let $f:P\righarrow G$ represent a section of AdP
%\end{proof}

\begin{prop}
Let $BG$ be the classifying space for $G$. Then in homotopy theory
\begin{equation*}
B\mathcal{G}(P)={\rm{Map}}_f(M,BG).
\end{equation*}
\end{prop}
The subscript $f$ denotes the component of Map$(M,BG)$ which contains the map $f$ that induces $P$. 
\begin{proof}
Let 
\begin{equation*}
G\rightarrow EG\rightarrow BG
\end{equation*}
be a universal bundle for $G$, and consider the space Map$(P,EG)$ of $G$-equivariant maps of $P$ to $EG$. The group $\mathcal G(P)$now acts naturally on this space by composition, to yield the principal fibring
\begin{equation*}
\mathcal G(P)\longrightarrow {\rm{Map}}(P,EG)\longrightarrow {\rm{Map}}_f(M,BG).
\end{equation*}
If $BG$ is paracompact and locally contractible, $\pi$ will be a locally trivial principal fibring. The total space Map$_f(P,EG)$ is contractible so that this is a universal bundle for $\mathcal G(P)$, and
\begin{equation*}
B\mathcal G(P)={\rm Map}_f(M,BG)
\end{equation*}
as was asserted. 

\end{proof}

\begin{equation}
\vee_{i=1}^t (S^3\wedge S^4)\vee(\vee_{j=1}^{d-t} P^4(p^r) 
\end{equation}

\chapter{Counting homotopy types of gauge groups}

\section{Samelson and Whitehead products}

Let $G$ be a homotopy associative $H$-space with multiplication $\mu:G\times G\rightarrow G$ and homotopy inverse $\iota:G\rightarrow G$. We write 
\begin{equation*}
\mu(x,y)=xy\quad {\rm{and}}\quad\iota(x)=x^{-1}
\end{equation*}
Then we write the commutator map [\;,\;]$:G\times G\rightarrow G$ as
\begin{equation*}
[x,y]=\mu(\mu(\mu(X,y),\iota(x)),\iota(y))=((xy)x^{-1})y^{-1}
\end{equation*}
or if we ignore the homotopy associativity, as
\begin{equation*}
[x,y]=xyx^{-1}y^{-1}
\end{equation*}

Since it is nullhomotopi con $G\vee G$ the commutator map factors as follows:
\begin{equation*}
G\times G\rightarrow G\wedge G\overset{\overline{[\;\;,\;\;]}}{\longrightarrow}G.
\end{equation*}

IF $f:X\rightarrow G$ and $g;Y\rightarrow G$ are maps, then the commutator
\begin{equation*}
C(f,g)=[\;,\;]\circ(f\times g)=fgf^{-1}g^{-1}:X\times Y\rightarrow G\times G\rightarrow G.
\end{equation*}
factors up to homotopy through the map
\begin{equation*}
[f,g]=\overline{[\;\;,\;\;]}\circ(f\wedge g):X\wedge Y\rightarrow G\wedge G\rightarrow G
\end{equation*}

\begin{defi} 
The map $[f,g]:X\wedge Y\rightarrow G$ is called the Samelson product of $f:X\rightarrow G$ and $g:Y\rightarrow G$.
\end{defi}
This map is well defined up to homotopy since the sequence of cofibrations 
\begin{equation*}
X\vee Y\rightarrow X\times Y\rightarrow X\wedge Y\rightarrow \Sigma X\vee\Sigma Y\rightarrow \Sigma(X\times Y)
\end{equation*}

\begin{prop}
The Samelson product vanishes in the range $G$ is homotopy commutative.
\end{prop}

Samelson products are natural with respect to maps $f_1:X_1\rightarrow X$, $g_1:Y_1\rightarrow Y$, and $H$-maps of $H$-spaces $\psi:G\rightarrow H$, that is 
\begin{equation*}
[\psi\circ f\circ f_1,\psi\circ g\circ g_1]\simeq \psi\circ [f, g]\circ(f_1\wedge g_1).
\end{equation*}
 
Given maps $\bar f:\Sigma X\rightarrow Z$, $\bar g:\Sigma Y\rightarrow Z$ with respective adjoints

\begin{equation*}
f:X\overset{\Sigma}{\rightarrow}\Omega\Sigma X\overset{\Omega\bar f}{\rightarrow}\Omega Z,\qquad g:X\overset{\Sigma}{\rightarrow}\Omega\Sigma X\overset{\Omega\bar g}{\rightarrow}\Omega Z
\end{equation*}

we define the Whitehead product $[\bar f,\bar g]_w$ to be the adjoint of the Samelson product $[f,g]$, namely, 
\begin{defi}
The Whitehead product $[\bar f,\bar g]_w$ is the compositon 
\begin{equation*}
\Sigma(X\wedge Y)\overset{\Sigma[f,g]}{\rightarrow}\Sigma\Omega Z\overset{e}{\rightarrow}Z
\end{equation*}
\end{defi}

As with Samelson products, Whitehead products are natural with respect to maps.

\begin{teo}[Crabb and Sutherland Ref] 
Let $K$ be a connected finite complex and let $G$ be a compact connected Lie group. As $P$ ranges over all principal $G$-bundles with base $K$, the number of homotopy types of $\mathcal{G}(P)$ is finite.
\end{teo}

\begin{prop}[Spreafico Ref]
The homotopy type of the gauge group of all the principal $SU(2)$-bundles over $S^n$, with $n=7,8$, is the same and is the one of the trivial bundle, namely $B\mathcal G(S^n)\sim\Omega^n_0SU(2)\times SU(2)$.
\end{prop}

\section{Poincare Duality}

In this section I will present the background to understand the construction of the topological spaces I will work with in the next sections. To doing so I will have to give a brief introduction to the concepts required to
state the Poincare Duality Theorem.

\begin{defi}
A manifold of dimension $n$ or an $n$-manifold is a Hausdorff space $M$ in which each point has an open neighbourhood homeomorphic to $\mathbb{R}^n$. 
\end{defi}

The dimension of $M$ is characterized by the fact that for $x\in M$, the homology group $H_i(M,M-\{x\})$ is nonzero only for $i=n$. A compact manifold is called closed  to distinguish it from the more general notion of a compact manifold with boundary. 

\begin{defi}
A local orientation of $M$ at a point $x$ is a choice of generator $\mu_x$ of the infinite cyclic group $H_n(M,M-\{x\})$. 
\end{defi}
To simplify notation I will write $H_n(X\mid A)$ for $H_n(X,X-A)$.

\begin{defi}
An orientation of an $n$-dimensional manifold $M$ is a function $x\mapsto\mu_x$ assigning to each $x\in M$ a local orientation $\mu_x\in H_n(M\mid x)$, satisfying the local consistency condition that each  $x\in M$ has a neighbourhood $\mathbb{R}^n\subset M$ containing an open ball $B$ of finite radius about $x$ such that all the local orientations $\mu_y$ at points $y\in B$ are the images of one generator $\mu_B$ of $H_n(M\mid B)\cong H_n(\mathbb{R}^n\mid B)$ under the natural maps $H_n(M\mid B)\rightarrow H_n(M\mid y)$. 
\end{defi}
If an orientation exists for $M$, then $M$ is called orientable. One can generalise the definition of orientation by replacing the coefficient group $\mathbb{Z}$ by any commutative ring $R$ with identity. The orientability of a closed manifold is reflected in the structure of its homology, according to the following results.

\begin{teo}
Let M ne a closed connected $n$-manifold. Then:
\begin{enumerate}[a)]
\item  If $M$ is orientable, the map $H_n(M;R)\rightarrow H_n(M\mid x;R)\cong R$ is an isomorphism for all $x\in M$.
\item If $M$ is not orientable, the map $H_n(M;R)\rightarrow H_n(M\mid x;R)\cong R$ is injective with image $\{r\in R\mid 2r=0\}$ for all $x\in M$.
\item $H_i(M;R)=0$ for $i>n$.
\end{enumerate}
\end{teo}



\begin{defi}
A fundamental class for a closed orientable manifold $M$ with coefficients in $R$ is an element of $H_n(M;R)$ whose image in $H_n(M\mid x;R)$ is a generator for all $x$.
\end{defi}
\begin{defi}
For an arbitrary space $X$ and coefficient ring $R$, define an $R$-bilinear cap product $\frown:C_k(X;R)\times C^l(X;R)\rightarrow C_{k-l}(X;R)$ for $k\geq l$ by setting
\begin{equation*}
\sigma\frown\varphi=\varphi(\sigma\mid[v_0,\dots,v_l])\sigma\mid[v_l,\dots,v_k]
\end{equation*}
for $\sigma:\Delta^k\rightarrow X$ and $\varphi\in C^l(X;R).$  
\end{defi}
\begin{teo}[Poincare duality theorem]
Let M be a closed and oriented $n$-manifold with fundamental class $[M]\in H_n(M;R)$. The map $D:H^k(M;R)\rightarrow H_{n-k}$ defined by
 \begin{equation*}
 D(\alpha)=[M]\frown\alpha
\end{equation*}
is an isomorphism for all $k$.
\end{teo}

\section{Classification of 2-connected 7-manifolds}

\begin{defi}
A homologically graded spectral sequence $E=\{E^r\}$ consists of a sequence of $\mathbb{Z}$-bigraded $R$ modules $E^r=\{E^r_{p,q}\}_{r\geq1}$ together with differentials 
\begin{equation*}
d^r:E^r_{p,q}\rightarrow E^r_{p-r,q+r-1}
\end{equation*}
such that $E^{r+1}\cong H_*(E^r)$. 
\end{defi}

\begin{defi}
A cohomologically graded spectral sequence $E=\{E_r\}$ consists of a sequence of $\mathbb{Z}$-bigraded $R$ modules $E_r=\{E_r^{p,q}\}_{r\geq1}$ together with differentials 
\begin{equation*}
d^r:E_r^{p,q}\rightarrow E_r^{p+r,q-r+1}
\end{equation*}
such that $E_{r+1}\cong H_*(E_r)$. 
\end{defi}

\begin{defi}
Let $D$ and $G$ be modules. An exact couple is an exact triangle
\begin{displaymath}
\xymatrix{ 
D \ar[rr]^i& &D\ar[ld]^{j} \\
 & G\ar[lu]^k &.
}
\end{displaymath}
\end{defi}

A spectral sequence $\{E_r\}$ is obtained from an exact couple by defining $E_1=G$, $d_1=jk$ and defining the derived couple
\begin{displaymath}
\xymatrix{ 
i(D) \ar[rr]^{i'}& &i(D)\ar[ld]^{j'} \\
 & E_2\ar[lu]^{k'} &.
}
\end{displaymath}
where $E_2=H_{*}(E_1)$ with respect to $d_1$, $i'$ is induced by $i$, $k'$ is induced by $k$, and $j'(i(a))=\{j(a)\}$. One can show these maps are well defined and the derived couple will again be exact couple. This process leads to an inductive definition of a spectral sequence. 

Let $C$ be a torsion-free chain complex over $\mathbb{Z}$. From the short exact sequence of groups
\begin{equation*}
0\longrightarrow\mathbb{Z}\longrightarrow\mathbb{Z}\longrightarrow\mathbb{Z}/p\mathbb{Z}
\end{equation*}
we obtain a short exact sequence of chain complexes
\begin{equation*}
0\longrightarrow C\longrightarrow C\longrightarrow C\otimes\mathbb{Z}/p\mathbb{Z}.
\end{equation*}

By the usual argument the homology of 
\begin{equation}
%\begin{displaymath}
\xymatrix{ 
H_{*}(C) \ar[rr]^{i_{*}}& &H_{*}(C)\ar[ld]^{j_{*}} \\
 & H_{*}(C\otimes\mathbb{Z}_p)\ar[lu]^{\partial_p} &.
}
%\end{displaymath}
\end{equation}

\begin{defi}
The spectral sequence associated with the exact couple (3.1) is called the Bockstein spectral sequence of $C$ $mod\;p$.
\end{defi}

Let $M$ be a closed 2-connected manifold. The non-zero homology groups of $M$ are $H_0(M)\cong\mathbb{Z}$, $H_3(M)$, $H_4(M)$, $H_7(M)\cong\mathbb{Z}$ where by duality $H_3(M)$ is free abelian of the same torsion-free rank as $H_3(M)$. The cohomology groups are given by Poincare duality, thus $G=H_3(M)\cong H^4(M)$.
Let $T$ denote the torsion group subgroup of $G$. Thus we have a nonsingular bilinear map
\begin{equation*}
b:T\times T\rightarrow S
\end{equation*}
where $S=\mathbb{Q}/ \mathbb{Z}$ defined as follows 



\section{Principal SU(n)-bundles over n-manifolds, n=7,8}

We start with analysing the case of a 8-manifold since the CW-complex structure is easier to handle. Analysing the 8-manifold case will be a first attempt to look into high dimensional manifolds. 







\end{document}