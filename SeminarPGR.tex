\documentclass{beamer} 
\input{xypic} 
\xyoption{color} 

%\usetheme{Singapore} 
\usetheme{CambridgeUS}
\usepackage{ bbold }

%breaksthrough proofs

\makeatletter
\newenvironment<>{proofs}[1][\proofname]{%
    \par
    \def\insertproofname{#1\@addpunct{.}}%
    \usebeamertemplate{proof begin}#2}
  {\usebeamertemplate{proof end}}
\makeatother


% expression abbreviations 
\newcommand{\hlgy}[1]{\ensuremath{H_{*}(#1)}}
\newcommand{\rhlgy}[1]{\ensuremath{\widetilde{H}_{*}(#1)}} 
\newcommand{\gk}{\ensuremath{\mathcal{G}_{k}}} 

% basic diagram commands
\newcommand{\namedright}[3]{\ensuremath{#1\stackrel{#2}
 {\longrightarrow}#3}}
\newcommand{\nameddright}[5]{\ensuremath{#1\stackrel{#2}
 {\longrightarrow}#3\stackrel{#4}{\longrightarrow}#5}}
\newcommand{\namedddright}[7]{\ensuremath{#1\stackrel{#2}
 {\longrightarrow}#3\stackrel{#4}{\longrightarrow}#5
  \stackrel{#6}{\longrightarrow}#7}}
\newtheorem{defi}{Definition}
\newtheorem{teo}{Theorem}
\newtheorem{prop}{Proposition}

\title{A topological three-dimensional quantum field theory of gravity.}  
\author{Ingrid Membrillo}
%\author{ Dr Stephen Theriault}
\date{December 2014} 

\begin{document} 

\begin{frame}
\titlepage
\end{frame} 

\begin{frame}\frametitle{Plan}
\begin{enumerate}
\item {Motivation and background}
\medskip
\item {Axiomatic formulation of a topological quantum field theory}
\medskip
\item {State sum invariants as a TQFT of gravity}
\medskip
%\item {Conclusions}
%\medskip
%\item
\end{enumerate}
\end{frame}

\begin{frame}\frametitle{1) Motivation and background} 

Quantum gravity (QG) seeks to describe gravitational field through the principles of quantum mechanics. 
\begin{figure}[H] 
\begin{center}
\includegraphics[width=0.35\textwidth]{QG1}
\end{center}
%\caption{(a) $X\vee Y$; (b) $\Sigma X=S^1\wedge X$.}
\end{figure}
\pause



%To find an adequate formulation of such a theory hasn't been an easy task. There are two big competitors at present:
%\begin{itemize}
%\medskip
%\item String theory
%\item Loop quantum gravity (Spinfoam as a covariant version)
%\end{itemize}

\end{frame}


\begin{frame}{1) Motivation and background}

\medskip

General relativity is the discovery that spacetime and the gravitational field are the same physical entity. All fields we know exhibit quantum properties at some scale, therefore it is believe spacetime to have quantum properties as well.
\begin{figure}[H] 
\begin{center}
\includegraphics[width=0.7\textwidth]{gr.jpg}
\end{center}
%\caption{(a) $X\vee Y$; (b) $\Sigma X=S^1\wedge X$.}
\end{figure}

%\begin{itemize}
%\item With Hilbert spaces and operators (hamiltonian)
%\item As as sum over paths (lagrangian)
%\end{itemize}

\end{frame}


\begin{frame}\frametitle{1) Motivation and background}

\begin{defi}
A \emph {differential manifold} of dimension $n$ is a topological space $M$ with a family of bijective maps $\phi_\alpha:U_\alpha\subset M\rightarrow V_\alpha\subset \mathbb R^n$ such that 
\medskip

\begin{enumerate}[i)]
\item $\underset{\alpha}{\bigcup}U_{\alpha}=M$
\item for all $\alpha, \beta$ such that $U_{\alpha\beta}:=U_\alpha\cap U_\beta\neq\emptyset$, the sets $\phi_\alpha(U_{\alpha\beta})$ and $\phi_\beta(U_{\alpha\beta})$ are open in $\mathbb{R}^n$ an the maps 
$\phi_\beta\circ\phi_{\alpha}^{-1}$ are smooths.
\end{enumerate}
\end{defi}
\medskip
\pause

The pairs $(U_\alpha, \phi_\alpha)$ are called coordinate charts. The set of all charts is called an atlas. We will also require that any two points have disjoint neighborhoods and the atlas of $M$ is countable. 


\end{frame} 

\begin{frame}\frametitle{1) Motivation and background}
\begin{defi}
Let $M$ and $N$ be differentiable manifolds. A continuous map $f:M\rightarrow N$ is called $differentiable$ in $p$ if for every chart $(V,\phi)$ with $f(p)\in V$ there exists a chart $(U,\psi)$ with $p\in U$ such that $f(U)\subset V$ and 
$\psi\circ f\circ\phi^{-1}$ is differentiable in $\phi(p)$.
\end{defi}
\pause

\begin{defi}
A differentiable bijective map $f:M\rightarrow N$ with $f^{-1}$ differentiable is called a \emph{diffeomorphism}.
\end{defi}
\end{frame}

\begin{frame}\frametitle{2) Axiomatic formulation of Topological Quantum Field Theory }

From the mathematical point of view we expect that a QG theory  keeps
\begin{itemize}
\item general features of quantum theory (probabilistic)
\item topological framework of spacetime

\end{itemize}

Both features are embraced by the precise notion of a topological quantum field theory.
\medskip

Atiyah (1988) gave a formal mathematical framework to develop quantum theories in physics.
\pause

\end{frame}

\begin{frame}\frametitle{2) Axiomatic formulation of Topological Quantum Field Theory}

\begin{defi}
A topological quantum field theory (TQFT) in dimension $d$ defined over a ground ring $R$, consist of the following data:
\medskip
\pause
\begin{enumerate}[A)]
\item A finitely generated  $R$-module $Z(\Sigma)$ associated to each oriented closed smooth $(d-1)$-dimensional manifold $\Sigma$.
\medskip
\pause
\item An element $Z(M)\in Z(\partial M)$ associated to each oriented smooth $d$-dimensional manifold (with boundary) $M$.
\end{enumerate}
\end{defi}
\pause

%$Z(M)$ is also called the \emph{partition function}.

\end{frame}

\begin{frame}\frametitle{2) Axiomatic formulation of Topological Quantum Field Theory}

These data are subject to the following axioms:
\medskip

\begin{enumerate}[1)]
\item $Z$ is functorial with respect to orientation preserving diffeomorphisms of $\Sigma$ and $M$. That is, given an orientation preserving diffeormorphism $f:\Sigma\rightarrow \Sigma'$ induces an isomorphism 
\begin{equation*}
Z(f):Z(\Sigma)\rightarrow Z(\Sigma')
\end{equation*}
and if $g:\Sigma'\rightarrow\Sigma''$ then
\begin{equation*}
Z(fg)=Z(f)Z(g)
\end{equation*}
\pause

If $f$ extends to an o. p. diff. $M\rightarrow M'$, with $\partial M=\Sigma$, $\partial M'=\Sigma'$, then $Z(f)$ take $Z(M)$ to $Z'(M)$.

\end{enumerate}
\end{frame}

\begin{frame}\frametitle{2) Axiomatic formulation of Topological Quantum Field Theory}
\begin{enumerate}[2)]
\item $Z(\Sigma^{*})=Z(\Sigma)^{*}$ where $\Sigma^{*}$ is $\Sigma$ with opposite orientation and $Z(\Sigma)^{*}$ denotes de dual module. 
Note that theres is nothing said about the orientation of $M$.
%\begin{itemize}
%\item When $R$ is a field, $Z(\Sigma)$ and $Z(\Sigma)^{*}$ are dual vector spaces.  
%\medskip 

%\item If R=$\mathbb{Z}$ the relation is like that between integer homology and cohomology
%\end{itemize}
\pause

\item [3)]$Z$ is multiplicative.
\end{enumerate}
Thus, for disjoint unions
\begin{equation*}
Z(\Sigma_1\cup\Sigma_2)=Z(\Sigma_1)\otimes Z(\Sigma_2).
\end{equation*}

\end{frame}

\begin{frame}\frametitle{2) Axiomatic formulation of Topological Quantum Field Theory}
Moreover if $\partial M_1=\Sigma_1\cup \Sigma_3$, $\partial M_2=\Sigma_2\cup \Sigma_3^{*}$ and $M=M_1\cup_{\Sigma_3}M_2$ is the manifold obtained by identification of the common $\Sigma_3-component$.
\begin{figure}[H] 
\begin{center}
\includegraphics[width=0.4\textwidth]{bordism}
\end{center}
%\caption{(a) $X\vee Y$; (b) $\Sigma X=S^1\wedge X$.}
\end{figure}
\pause

Then we require
\begin{equation*}
Z(M)=<Z(M_1),Z(M_2)>
\end{equation*}
where $<,>$ denotes the natural pairing

\begin{equation*}
Z(\Sigma_1)\otimes Z(\Sigma_3)\otimes Z(\Sigma_3)^{*}\otimes Z(\Sigma_2)\rightarrow Z(\Sigma_1)\otimes Z(\Sigma_2)
\end{equation*}

\end{frame}

\begin{frame}\frametitle{2) Axiomatic formulation of Topological Quantum Field Theory}

The multiplicative axiom shows that when $\Sigma=\emptyset$ the vector space $Z(\Sigma)$ is idempotent. So therefore it is zero or isomorphic to the ground ring $R$. 
\medskip
\pause

\begin{itemize}
\item [4a)*] $Z(\emptyset)=R$ when $\Sigma$ is the empty $(d-1)$-dimensional manifold.
\end {itemize}
\medskip
\pause

Similarly  when $M=\emptyset$, $Z(M)\in R$ is an idempotent element.
\begin{itemize}
\item [4b)*] $Z(\emptyset)=1$ when $M$ is the empty $d$-dimensional manifold.
\end {itemize}
\medskip
\pause
Note that when $M$ is a closed $d$-dimensional manifold so that $\partial M=\emptyset$, then
\begin{equation*}
Z(M)\in Z(\emptyset)=R
\end{equation*}

\end{frame}

\begin{frame}\frametitle{3) State sum invariants as a TQFT of gravity}

Einstein equations (movement equations) can be derived through minimizing a function known as the action \emph{S}:
\medskip

\begin{equation*}
S= \frac{1}{16\pi}\int R\sqrt{det(g_{\mu\nu)}}d^4x
\end{equation*}
\medskip
\pause

1961 Regge gave a discrete formulation of general relativity using triangulated manifolds. 
\begin{figure}[H] 
\begin{center}
\includegraphics[width=0.35\textwidth]{regge}
\end{center}
%\caption{(a) $X\vee Y$; (b) $\Sigma X=S^1\wedge X$.}
\end{figure}

\end{frame}

\begin{frame}\frametitle{3) State sum invariants as a TQFT of gravity}
Regge showed that the  curvature of a triangulated n dimensional manifold is encoded in the $(n-2)$ skeleton. 
\begin{figure}[H] 
\begin{center}
\includegraphics[width=0.35\textwidth]{pt}
\end{center}
%\caption{(a) $X\vee Y$; (b) $\Sigma X=S^1\wedge X$.}
\end{figure}


The discrete form of the action S from which Einstein field equations are derived is given by 

\begin{equation*}
S_R=\sum\left|\sigma^{i}(T)\right|\varepsilon_{i}
%S_R=\sum\frac{\partial\left|\sigma^{i}\right|}{\partial l_{j}}\varepsilon_{i}=0
\end{equation*}


\end{frame}

\begin{frame}\frametitle{3) State sum invariants as a TQFT of gravity}
In quantum mechanics the action is replaced by a partition function Z:
\begin{equation*}
Z=\int e^{iS}D(x)
%S_R=\sum\frac{\partial\left|\sigma^{i}\right|}{\partial l_{j}}\varepsilon_{i}=0
\end{equation*}

Given a manifold M, Z(M) obtained from the TQFT will have the physical meaning of the partition function. 

\end{frame}

%\begin{frame}\frametitle{State sum invariants}

%For each triangulated closed surface F we define a $K$-module C(F) to be the module freely generated over $K$ by admissible colorings of $F$.
%If $F=\emptyset$, then we put C(F)=K.  

%\end{frame}

%\begin{frame}\frametitle{The James construction}


%\end{frame}




\end{document}



