\documentclass[A4,12pt,oneside]{book}
%\usepackage[margin=0.5in]{geometry}
\textheight=19.94cm 
\textwidth=14.5cm 
%\topmargin=-4cm 
%\bottomargin=-2cm 
%\oddsidemargin=-0.2 cm 
%\evensidemargin=2 cm 
%\renewcommand{\baselinestretch}{1.5} 
%\parindent=1.5cm 
\usepackage{tikz}
\usepackage{textcomp}
\usetikzlibrary{matrix,arrows,decorations.pathmorphing}
\usepackage{amsmath,amscd}
\usepackage{amssymb}
\usepackage{amsthm}
\usepackage[all,cmtip]{xy}

\usepackage[toc,page]{appendix}

%\usepackage[spanish,es-lcroman]{babel}
%\usepackage[english, es-lcroman]{babel}

\usepackage{courier}

\usepackage{dsfont}

\usepackage{enumerate}
\usepackage{esint}

\usepackage{float}
\usepackage[T1]{fontenc}

\usepackage{graphicx}

\usepackage[utf8x]{inputenc}

\usepackage{listings}

\usepackage{mathrsfs}
\usepackage{mathtools}

\usepackage{stmaryrd}

\usepackage{textcomp}
\usepackage{pdfpages}

\pagestyle{headings}

\theoremstyle{definition}

\newtheorem{teo}{Theorem}
\newtheorem{defi}{Definition}
\newtheorem{ejem}{Examples}
\newtheorem{lema}{Lemma}
\newtheorem{prop}{Proposition}
\newtheorem{demo}{Proof}

\begin{document}

% \addcontentsline{toc}{chapter}{Introducci\'on}{}
        \newpage

%  \includepdf[pages={1}]{cover.pdf}

\chapter{Steenrod algebra (Masher and Tangora)}

\section{Introduction}
%{\LARGE\textbf{Steenrod algebra (Masher and Tangora)}}\\

A cohomology operation is a natural transformation $\theta:H^m(X;G)\rightarrow H^n(X;H)$, that is for all maps $f:X\rightarrow Y$ the following diagram
\begin{displaymath}
\xymatrix{ 
H^m(X;G)\ar[r]^{\theta_X}\ar[d]^{f^*} &H^m(X;G')\ar[d]^{f^*} \\
H^m(Y;G)\ar[r]^{\theta_Y} &H^m(Y;G')
}
\end{displaymath}
commutes.
\begin{defi}
We denote by $K(G, n)$ any space which has only one non-trivial homotopy group, namely $\pi_n(K(G,n))=G$. 
\end{defi}
The Hurewicz homomorphism $h:\pi_i(X)\rightarrow H_i(X)$ is defined for any $X$ and any $i$ by choosing a generator $u$ of $H_i(S^i)$ and putting $h:[h]\rightarrow f_*(u)$ where $f:S^i\rightarrow X$. 

\begin{defi}
A space $X$ is said to be $n$-connected if $\pi_i(X)$ is trivial for all $i\leq n$. 
\end{defi} 
The Hurewicz theorem states that if $X$ is $(n-1)$-connected, then the Hurewicz homomorphism is an isomorphism in dimensions $ i \leq n$ and is still an epimorphism in dimension $n+1$. This theorem is modified if
$n=1$; in this case the epimorphism $h:\pi_1(X)\rightarrow H_1(X)$ has a kernel the commutator subgroup of $\pi(X)$ and $h$ does not necessarily maps $\pi_2(X)$ onto $H_2(X)$.\\
The universal coefficient theorem for cohomology gives an exact sequence
\begin{equation*}
0\rightarrow {\rm{Ex}}t(H_{n-1}(X),G)\rightarrow H^n(X;G)\rightarrow {\rm{Hom}}(H_n(X),G)\rightarrow 0
\end{equation*}
for any space X. If $X$ is $(n-1)$-connected, this becomes an isomorphism between the last two terms of the sequence, since Ext$(0,G)=0$. Now if $G=\pi_n(X)$ then the group Hom$(H_n(X),\pi_n(X))$ contains $h^{-1}$, the inverse of the Hurewicz homomorphism, which is also an isomorphism.
\begin{defi}
Let $X$ be $(n-1)$-connected. The fundamental class of $X$ is the cohomology class $\iota\in H^n(X;\pi_n(X))$ which correspond to $h^{-1}$ under the above isomorphism.We write sometimes $\iota_n$ or $\iota_X$ instead. 
\end{defi} 
\begin{teo}
There is a one-to-one correspondence $[X,K(G,n)]\rightarrow H^n(X;G)$, given by $[f]\rightarrow f^*(\iota_n)$.
\end {teo}
Notation: let's write $H^m(G,n;G')$ for $H^m(K(G,n);G')$. 

\begin{teo}
Let denote $O(n,G; m,G')$ the set of all cohomolgy operations. There is a one-to one correspondence 
\begin{equation*}
O(G,n;G',m)\rightarrow H^m(G,n;G')
\end{equation*}
given by $\theta\rightarrow\theta(\iota_n)$.
\end{teo}

Read about Obstruction theory!!!


\section{Steenrod squares}

Steenrod squares are cohomology operation of type $(\mathbb {Z}_2,n;\mathbb{Z}_2,n-i)$. Given an abelian group $G$ it is possible to construct a $CW$-complex $K(G,n)$ where $n\geq 2$. In particular, we can construct a complex $K(\mathbb {Z}_2,1)$.

\begin{defi}
For each integer $i\geq 0$, define a \emph{cup-i product}
\begin{equation*}
C^p(K)\otimes C^q\rightarrow C^{p+q-i}(K):(u,v)\rightarrow u\smile_i v
\end{equation*}
\end{defi}
 where $C^p(X)$.

\chapter{Fibre bundles and Classifying Spaces}

\begin{defi}
Let $B$ be a pointed topological space. A (locally trivial) fibre bundle over $B$ consists of a map $p:E\rightarrow B$ such that for all $b\in B$ there exists an open neighbourhood $U$ of $b$ for which there is a homeomorphism $\phi:p^{-1}(U)\rightarrow p^{-1}\times U$ satisfying 
\begin{equation*}
\pi''\circ\phi=p\bigl |_U
\end{equation*}
where $\pi''$ denotes projection onto the second factor. 
\end{defi}

\begin{defi}
Let $G$ be a topological group $B$ and $B$ a topological space. A \emph{principal G-bundle} over $B$ consists of a fibre bundle $p:X\rightarrow B$ together with an action $G\times X\rightarrow X$ such that:
\begin{itemize}
\item the  map $G\times X\rightarrow X\times X$ given by $(g,x)\mapsto (x,g\cdot x)$ maps $G\times X$ homeomorphically to its image;
\item $B= X/G$ and $p:X\rightarrow X/G is the quotient map;

\end{itemize}
\end{defi}














\end{document}